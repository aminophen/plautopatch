%# -*- coding: utf-8 -*-
\ifx\epTeXinputencoding\undefined\else % defined in e-pTeX (> TL2016)
  \epTeXinputencoding utf8    % ensure utf-8 encoding for platex
\fi

\documentclass[a4paper]{jsarticle}
\usepackage{doc}
\usepackage{plautopatch}% load this here to make \GetFileInfo work
\GetFileInfo{plautopatch.sty}
\title{Package \textsf{plautopatch} \fileversion}
\author{Hironobu Yamashita}
\date{\filedate}
\begin{document}

\maketitle

日本のp\LaTeX/up\LaTeX{}フォーマットや専用パッケージが、
これらを知らない\LaTeX{}パッケージ(しばしば海外で作られた
汎用のもの)と衝突することがあります。
最悪の場合にはエラーが出たり、誤った出力が得られたり
することがあります。

この\textsf{plautopatch}の目的は、こうした非互換を意識せずに
済むようにすることです。具体的には、p\LaTeX/up\LaTeX{}と
衝突するパッケージが使われた場合に、その衝突を解消するパッチを
提供するパッケージを必要に応じて自動的に読み込みます。
こうすることで、ソースコードを簡潔にできるだけでなく、
p\LaTeX/up\LaTeX{}で動作するソースと通常の\LaTeX{}ソースの
見た目を近づけることができます。

このパッケージはGitHubで開発しています。
\begin{verbatim}
    https://github.com/aminophen/plautopatch
\end{verbatim}

\section*{動作条件}

このパッケージは
\textsf{filehook}パッケージ(Martin Scharrer氏の作)に依存します。

\section*{使い方}

このパッケージを\LaTeX{}ソースの冒頭で読み込みます。
このために、\verb+\RequirePackage{plautopatch}+を
\verb+\documentclass+や他のコマンドよりも\emph{前}に
読み込むことをお勧めします
(クラスファイルなどが問題のあるパッケージを読み込む可能性もあるため)。

例を示します。
\begin{verbatim}
  %\RequirePackage{plautopatch}
  \documentclass{tarticle}% 縦組クラス(plext使用)
  \usepackage{array}% plextと非互換
  \begin{document}
  ...
  \end{document}
\end{verbatim}
上記の例では、
\textsf{tarticle}クラスが内部で読み込む\textsf{plext}パッケージと、
ソース中で\verb+\usepackage+している\textsf{array}パッケージが
衝突してエラーになる場合があります。しかし、冒頭で
\verb+\RequirePackage{plautopatch}+とだけ書いておけば、
\textsf{array}パッケージの時点で\textsf{plextarray}パッケージが
追加で読み込まれるため、問題が解消します。
このように自動追加されたパッケージは、\verb+\end{document}+の
時点で次のように一覧として表示されます(複数の場合は
コンマと空白で区切ったリストになります)。
\begin{verbatim}
***** List of packages loaded by `plautopatch': *****
 plextarray.
*****************************************************
\end{verbatim}

\section*{現在対応しているパッケージの一覧}

\def\ITEMoTo#1 (#2) -> #3 (#4){%
  \item \textsf{#1} (\textsf{#2})\\
    $\rightarrow$ \textsf{#3} (\textsf{#4})}
\def\ITEMoTx#1 (#2) -> #3 {%
  \item \textsf{#1} (\textsf{#2})\\
    $\rightarrow$ \textsf{#3} }
\def\ITEMxTx#1      -> #2 {%
  \item \textsf{#1}\\
    $\rightarrow$ \textsf{#2} }
\def\ITEMooTo#1 (#2) + #3 (#4) -> #5 (#6){%
  \item \textsf{#1} (\textsf{#2}) + \textsf{#3} (\textsf{#4})\\
    $\rightarrow$ \textsf{#5} (\textsf{#6})}
\def\ITEMxoTo#1      + #2 (#3) -> #4 (#5){%
  \item \textsf{#1}               + \textsf{#2} (\textsf{#3})\\
    $\rightarrow$ \textsf{#4} (\textsf{#5})}
\def\ITEMxoTx#1      + #2 (#3) -> #4 {%
  \item \textsf{#1}               + \textsf{#2} (\textsf{#3})\\
    $\rightarrow$ \textsf{#4} }

凡例:
\begin{itemize}
\item \texttt{<元のパッケージ>} (\texttt{<元が含まれるバンドル名>})\\
  \texttt{<パッチのパッケージ>} (\texttt{<パッチが含まれるバンドル名>})
\end{itemize}

現在のバージョン (\filedate\space\fileversion) がサポートしているのは
下記のパッケージです。
\begin{itemize}
\ITEMoTo  doc (latex) -> pldocverb (maintained here!)
\ITEMoTo  tracefnt (latex) -> ptrace/uptrace (platex/uplatex)
\ITEMoTo  fltrace (latex) -> pfltrace (platex)
\ITEMoTo  array (latex-tools) -> plarray (platex-tools)
\ITEMooTo array (latex-tools) + plext (platex) -> plextarray (platex-tools)
\ITEMooTo delarray (latex-tools) + plext (platex) -> plextdelarray (platex-tools)
\ITEMxoTx colortbl + plext (platex) -> plextcolortbl (maintained here!)
\ITEMxTx  arydshln -> plarydshln (maintained here!)
\ITEMxoTx arydshln + plext (platex) -> plextarydshln (maintained here!)
\ITEMxTx  siunitx -> plsiunitx (maintained here!)
\ITEMoTo  everysel (ms) -> pxeverysel (platex-tools)
\ITEMoTo  everyshi (ms) -> pxeveryshi (platex-tools)
\ITEMoTo  atbegshi (oberdiek) -> pxatbegshi (platex-tools)
\ITEMoTo  ftnright (latex-tools) -> pxftnright (platex-tools)
\ITEMoTo  multicol (latex-tools) -> pxmulticol (platex-tools)
\ITEMoTo  xspace (latex-tools) -> pxxspace (platex-tools)
\ITEMoTo  textpos (latex-tools) -> pxtextpos (platex-tools)
\ITEMxTx  pdfpages -> pxpdfpages (maintained here!)
\ITEMoTo  stfloats (sttools) -> pxstfloats (pxsttools)
\ITEMoTx  pgfrcs (pgf) -> pxpgfrcs (maintained here!)
\ITEMoTx  pgfcore (pgf) -> pxpgfmark (by Takayuki YATO)
\end{itemize}
もちろん、このリストは随時、追加・削除・置き換えていく予定です。
互換性の問題や追加したいパッケージがある場合はご一報ください。

\section*{特定のパッケージを除外したい場合}

デフォルトでは、上記のリストに登録されている\texttt{<元のパッケージ>}が
使われたことを検出すると、全て自動的にパッチを読み込みます。
しかし、時にはこれが逆効果となり、問題が起きる可能性は否定できません。
そのような場合は
\begin{verbatim}
    \plautopatchdisable{<元のパッケージ>}
\end{verbatim}
と書くことで、そのパッケージを検出対象から除外します。複数ある場合は
\begin{verbatim}
    \plautopatchdisable{<元のパッケージ1>,<元のパッケージ2>}
\end{verbatim}
のようにコンマで区切っていくつでも除外できます。

\section*{変更履歴}

\begin{itemize}
  \item 2018/08/21 v0.2  最初のCTANリリース版
  \item 2018/08/22 v0.3  元パッケージ検出の改良
  \item 2018/09/21 v0.5  \textsf{colortbl}と\textsf{pgf}もサポート
  \item 2018/10/02 v0.6  \textsf{arydshln}のサポート
  \item 2018/10/27 v0.8  \textsf{everysel}サポートの改良
  \item 2018/11/03 v0.9  \textsf{siunitx}のサポート
  \item 2018/11/25 v0.9b \textsf{multicol}と\textsf{doc}のサポート
  \item 2019/06/06 v0.9c \textsf{siunitx}のパッチ改良
  \item 2019/09/05 v0.9d \textsf{xspace}と\textsf{stfloats}のサポート
  \item 2020/02/25 v0.9e \textsf{textpos}のサポート
\end{itemize}

\end{document}
