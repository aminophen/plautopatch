\RequirePackage{plautopatch}

%%%%% VERBOSE %%%%%
% [TODO]
% * BEGIN OF FTNRIGHT が最後に一回余分に出ていて END と数が合わない
% * Begin Of Files と End Of Files が最後に一組多く出る
\AtBeginOfPackageFile{ftnright}{\typeout{BEGIN OF FTNRIGHT}}
\AtEndOfPackageFile{ftnright}{\typeout{END OF FTNRIGHT}}
\AtBeginOfPackageFile{pxftnright}{\typeout{BEGIN OF PXFTNRIGHT}}
\AtEndOfPackageFile{pxftnright}{\typeout{END OF PXFTNRIGHT}}
\AtBeginOfFiles{\typeout{Begin Of Files}}
\AtEndOfFiles{\typeout{End Of Files}}
%%%%%

\documentclass[a4paper,twocolumn]{tarticle}

%% Test: tracefnt
\usepackage{tracefnt}

%% Test: delarray
\usepackage{delarray}
%%

%% Test: ftnright
\usepackage{ftnright}
\def\hoge{ほげほげ。ふがふがふが。ぴよぴよぴよぴよ。} % ダミーテキスト用
\def\HOGE{\hoge\hoge\hoge\hoge\hoge\hoge\hoge\hoge} % 長めのダミーテキストを準備
%%

\begin{document}

[tracefnt]

\HOGE {\Large\hoge}

[ftnright]

\HOGE 脚注を\footnote{脚注です。}入れます。\HOGE

\HOGE もう一つ脚注を\footnote{ふたつめの脚注です。}入れます。

\begin{table}[b]
  \begin{tabular}{ll}
    \hline
    AAA & BBB \\
    CCC & DDD \\
    \hline
  \end{tabular}
\end{table}

\HOGE さらに脚注を\footnote{みっつめの脚注です。}入れます。
\HOGE

[delarray]

\[
  \begin{array}<t>[t]\{{c}\}
    3 \\ 4 \\ 5  \end{array}
  \begin{array}<t>[c]\{{c}\}
    2 \\ 3 \\ 4  \end{array}
  \begin{array}<t>[b]\{{c}\}
    1 \\ 2 \\ 3  \end{array}
\]
\[
  \newcolumntype{L}{>{$}l<{$}}
  f(x)=
    \begin{array}<t>\{{lL}.
      0 & if $x=0$ \\
      \sin(x)/x & otherwise
    \end{array}
\]

\end{document}
